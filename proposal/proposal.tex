\documentclass{article}      % Specifies the document class

\usepackage{float}
\usepackage{fullpage}
\usepackage[margin=0.5in]{geometry}
\usepackage{caption}
\usepackage{subcaption}
\usepackage{graphicx}
\usepackage{natbib}
\DeclareGraphicsExtensions{.pdf,.png,.jpg}
\graphicspath{{figures/}}

\title{Personalized Web Search \\ CS578: Project Proposal}  
\author{Asish Ghoshal, Anantha Raghuraman}      

\newcommand{\ip}[2]{(#1, #2)}
\newcommand{\iref}[1]{Fig.~\ref{#1}}
\newcommand{\xcite}[1]{(\cite{#1})}

%\newcommand{\ip}[2]{\langle #1 | #2\rangle}
                             % This is an alternative definition of
                             % \ip that is commented out.

\begin{document}             % End of preamble and beginning of text.

\maketitle                   % Produces the title.

This project aims to understand current literature in personalized web search and develop a novel approach to 
re-rank URLs of each Search Engine Results Page (SERP) returned by the search engine according to the personal preferences of the users. 
The objective is to personalize search using the long-term and short-term user context where long-term context is based on overall user history
while the short-term context is session based. 
Specifically, the algorithm would re-rank top-10 URLs returned by the search engine in response
to a user query using the history of clicks on URLs for all users and in particular, the user issuing the current query. 
The problem statement and data has been taken from the personalized web search challenge on kaggle.com.  

The evaluation is based on a variant of a dwell-time based model of personal relevance,
and is used in the state of the art research on personalized search \xcite{Eickhoff2013, Shokouhi2013, Wang2013}. The dwell time 
is defined as the elapsed time between the click on the document and the next click or the next query and is dwell time is well correlated 
with the probability of the user to satisfy his/her information need with the clicked document.
The Normalized Discounted Cumulative Gain (NDCG) measure will be the metric used to evaluate the performance of the algorithm.



\bibliographystyle{unsrtnat}
\bibliography{proposal}
\end{document}               % End of document.
